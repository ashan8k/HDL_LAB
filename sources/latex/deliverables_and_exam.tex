\newpage
\section{Deliverables}
Make a folder tree as indicated below and place your files and your 1 page long short report named 
\begin{center}
\textbf{hdllab2017\_summary\_xx.pdf}
\end{center}
 in the corresponding sub-folder. Pack everything in a zip-file named 
\begin{center}\textbf{hdllab2017\_xx.zip}
\end{center}
 (xx is your group number) and email it to 
\begin{center}
\href{mailto:sarath.mohanan@ies.tu-darmstadt.de}{sarath.mohanan@ies.tu-darmstadt.de}\\
by\\
11.08.2017 (Friday) 23:59:59
\end{center}
In the short report indicate the major milestones achieved and status of your processor with respect to the test cases provided.

Make the detailed main report named 
\begin{center}
\textbf{hdllab2017\_report\_xx.pdf}
\end{center}
 and email it by 
\begin{center}

16.08.2017 (Wednesday) 23:59:59
\end{center}
\begin{enumerate}
	\item Folder structure
	\begin{itemize}
	    \item \textbf{/designs}		(gate-level netlists, .sdf timing anotation)
		\item \textbf{/documents}					(put your PDF report here)
		\item \textbf{/reports}						(synthesis reports: area, power, timing, resources, references)
		\item \textbf{/sources/rtl}					(RTL Verilog source code)
		\item \textbf{/source/scripts/simulation}			(simulation scripts)
		\item \textbf{/source/scripts/synthesis}			(synthesis scripts)
		\item \textbf{/source/sim}			       (Modlesim project directory)
		\item \textbf{/source/software}				(makefile, c-source files and binaries)
		\item \textbf{/source/syn}			       (Synthesis directory)
		\item \textbf{/source/testbench} 		   (testbench source code)
		\item \textbf{/stimulus}					(test program binaries)
		\end{itemize}
	
	\item Report (12 pages maximum, English)
		\begin{itemize}
		\item General outline
			\begin{itemize}		 
			\item Introduction / motivation / goals / work distribution (1 page)
			\item Implementation / technical work (10 pages)
				\begin{itemize}
				\item Design
					\begin{itemize}
					\item figure: block diagram of processor architecture
					\item Discussion of processor architecture
					\end{itemize}					
				\item RTL Verification
					\begin{itemize}
					\item figure: block diagram of testbench
					\item Discussion of testbench
					\end{itemize}		
				\item Synthesis
					\begin{itemize}
					\item figure: top level schematic
					\item Discussion of schematic
					\item figure: critical path schematic 
					\item Discussion of critical path 
					\item Discussion of resources and resource sharing
					\end{itemize}				
				\item GL Verification
				\item (other, e.g. comparison between different architectures... you can go crazy here)
				\end{itemize}		 
			\item Evaluation / conclusion (1 page)
				\begin{itemize}
				\item 
				\end{itemize}
			\item Appendix
				\begin{itemize}
				\item timing report
				\item resource report
				\item area report
				\item power report
				\end{itemize}
			\end{itemize}
		\item Hints
			\begin{itemize}
			\item There is no cookbook recipe (and only few mathematical formulas) for processor design, only good/bad experience. Therefore reports and articles in this domain need to focus on the reasoning, i.e. why a particular design was chosen. Sections on design iterations, evaluation strategies, comparisons between different options/configurations and so on are most interesting to read. 
			\item Use passive mode: ''The registers are comprised of d-flipflops'' instead of  ''we used d-flipflops for the registers''. In the work distribution (beginning) and conclusion (very end) sections you may use active mode, i.e. ''Sarath was responsible for the decode stage'' or ''the authors will continue verification after tape-out''.
			\item Express yourself clearly and in a concise form.					
			\item This is a good occasion to learn \LaTeX, because you will need it for your thesis. After the lab you are given sufficient days to write your report for exactly this reason.
			\end{itemize}
		\item Sad but true
		\begin{itemize}
			\item Plagiarism is embarrassing! Point out your own as well as others' original work. Cite when you use other people's results (even when paraphrasing). Ask if in doubt. All reports and source code at IES are checked with automated tools.
			\end{itemize}
	\end{itemize}
\end{enumerate}

\newpage
\section{Examination - HDL Lab}
The examination is mandatory, oral and takes place in groups of 4. It lasts roughly 30-45 minutes and the date is by agreement, preferably in the week after report submission. Use the following collection of questions for preparation. Read ``explain'' as ``tell a non-technical person how it works.'' If you are able to do so, then you can also explain it to your examinor.

\begin{enumerate}
		\item Know your group's design
			\begin{itemize}			
			\item Explain your general architecture. (What where your considerations when chosing this particular one?)
			\item Explain some line of your entire source code.
			\item Explain something that you have written (or omitted) in your report.
			\item Explain how an instruction passes through your design.
			\item Which is the most expensive operation in terms of time/area/power.
			\end{itemize}		
		\item Verilog
			\begin{itemize}			
			\item Write code for: DFF, sensitive to rising/falling edge with/without enable, with (a)synchronous reset
			\item Write code for: purely combinational barrel shifter that shifts din 0, 1, 2 or 3 bits right and drives dout with the result. Empty bits are filled with zeros. selection is based on signal ''shift'' (00, 01, 10, 11). din/dout have a width of 8.
			\item Write code for: some other simple logic/FSM ...	
			\end{itemize}	
		\item Digital Design
			\begin{itemize}
			\item Explain: Moore, Mealy, latched-Mealy
			\item Explain: 2's Complement, signed and unsigned representaion, carry and overflow
			\item Explain: combinational and non-combinational
			\item Explain: D-FF and Latch.
			\item Explain: setup time and hold time
			\item Explain: gtech library, synthetic library, Design Ware Building Blocks library
			\item Explain: critical path
			\item Explain: maximum frequency
			\item Explain: the power report
			\item Explain: the timing report
			\item Explain: the area report
			\item Explain: the reference report
			\end{itemize}
		\item Pipelining
			\begin{itemize}			
			\item What is the idea behind pipelining?
			\item Explain: hazard
			\item What types of hazards do you know? When do they occur? How can you resolve hazards?
			\item What is a balanced pipeline?
			\end{itemize}
		\item Other questions related to anything you did in the lab.
\end{enumerate}

